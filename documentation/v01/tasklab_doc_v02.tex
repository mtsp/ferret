%%%%%%%%%%%%%%%%%%%%%%%%%%%%%%%%%%%%%%%%%
% Vertical Line Title Page 
% LaTeX Template
% Version 1.0 (27/12/12)
%
% This template has been downloaded from:
% http://www.LaTeXTemplates.com
%
% Original author:
% Peter Wilson (herries.press@earthlink.net)
%
% License:
% CC BY-NC-SA 3.0 (http://creativecommons.org/licenses/by-nc-sa/3.0/)
% 
% Instructions for using this template:
% This title page compiles as is. If you wish to include this title page in 
% another document, you will need to copy everything before 
% \begin{document} into the preamble of your document. The title page is
% then included using \titleGM within your document.
%
%%%%%%%%%%%%%%%%%%%%%%%%%%%%%%%%%%%%%%%%%

%----------------------------------------------------------------------------------------
%	PACKAGES AND OTHER DOCUMENT CONFIGURATIONS
%----------------------------------------------------------------------------------------

% To turn comments OFF simply comment out the \Commentstrue line
%\usepackage{xcolor}
\newif\ifComments
\Commentstrue

\ifComments
\newcommand{\chek}[1]{\noindent\textcolor{red}{Check: {#1}}}
\newcommand{\isadora}[1]{\noindent\textcolor{violet}{Isadora: {#1}}}
\newcommand{\guido}[1]{\noindent\textcolor{magenta}{Guido: {#1}}}
\newcommand{\joao}[1]{\noindent\textcolor{brown}{Joao: {#1}}}
\newcommand{\del}[1]{\noindent\textcolor{gray}{Removed: {#1}}}
\newcommand{\new}[1]{\noindent\textcolor{blue}{ {#1}}}
\newcommand{\ed}[1]{\noindent\textcolor{red}{ {#1}}}
\else
\newcommand{\chek}[1]{}
\newcommand{\isadora}[1]{}
\newcommand{\guido}[1]{}
\newcommand{\joao}[1]{}
\newcommand{\del}[1]{}
\newcommand{\new}[1]{#1}
\newcommand{\ed}[1]{#1}
\fi

\documentclass{article}

\usepackage{xcolor}
\usepackage{listings}
\usepackage{epsfig}
\usepackage{graphics, float}
\usepackage{color, soul}
\usepackage{multirow}

\newcommand*{\plogo}{\fbox{$\mathcal{PL}$}} % Generic publisher logo
\newcommand{\hlc}[2][yellow]{ {\sethlcolor{#1} \hl{#2}} }

%----------------------------------------------------------------------------------------
%	TITLE PAGE
%----------------------------------------------------------------------------------------

\newcommand*{\titleGM}{\begingroup % Create the command for including the title page in the document
\hbox{ % Horizontal box
	\hspace*{0.2\textwidth} % Whitespace to the left of the title page
	\rule{1pt}{\textheight} % Vertical line
	\hspace*{0.05\textwidth} % Whitespace between the vertical line and title page text
	\parbox[b]{0.85\textwidth}{ % Paragraph box which restricts text to less than the width of the page
		
		{\noindent\Huge\bfseries TaskLab Library }\\[2\baselineskip] % Title
		{\large \textit{API Specification \\ Release v0.1}}\\[4\baselineskip] % Tagline or further description
		{\Large \textsc{Unicamp Team}\\} % Author name
		{\date{}}
		
		\vspace{0.5\textheight} % Whitespace between the title block and the publisher
		%{\noindent The Publisher \plogo}\\[\baselineskip] % Publisher and logo
	}}
	\endgroup}

\newcommand{\colorbitbox}[3]{%
	\rlap{\bitbox{#2}{\color{#1}\rule{\width}{\height}}}%
	\bitbox{#2}{#3}}

%----------------------------------------------------------------------------------------
%	BLANK DOCUMENT
%----------------------------------------------------------------------------------------

\begin{document}

\pagestyle{empty} % Removes page numbers

\titleGM % This command includes the title page

\section{Introduction} \label{sec:int}
TaskLab was designed to enable the simulation of task parallelism applications without relying on multiple benchmarks. It  is connected to the task runtime (e.g. MTSP in Figure \ref{fig:arch}) through a set of functions calls which are described in the sections below. 

TasklLab can be useful  to perform a number of jobs related to task parallelism like: (a) Measure the performance of a given runtime and compare it with another libraries; (b) Check for correctness in the execution of applications in a given platform; and (c) Allow multiple benchmarking without relying in real applications. To enable such jobs TaskLab offers the following functions to the runtime designer:
\begin{itemize}
\item Generate, by the user parameters, a directed acyclic graph (DAG) that would represent an arbitrary application with task parallelism;
\item Serialize the generated DAG, allowing it to be reproduced repeatedly;
\item Visualize the simulated application by a \textit{.dot} file;
\item Communicate directly with the runtime in order to execute the generated DAG file;
\item Faster and practical performance analysis.
\end{itemize}

\begin{figure}[ht]
  \centering
  \includegraphics[width=13cm]{figures/MTSP-Driver.pdf}
  \caption{Architecture of the MTSP runtime and its interface to the TGA  driver and hardware module.}
  \label{fig:arch}
\end{figure}

\section{TaskLab API}
\guido{Let us use only a single include file tasklab.h; please adapt the text below and the code accordingly.  \\}

The following TaskLab API describes version 0.1. It consists of a single file, \textbf{TaskLab.h}, which contains the DAG generation and the dispatcher to the runtime, along with methods regarding visualization and serialization.

\guido{\\ Now that we have the library+driver defined with the proper definitions for tasks and deps, we should connect both}

\isadora{The current definition of tasks+deps are very important for the graph generation and validation, should we necessarily use the same definitions for connecting both of them? }

\subsection{TaskLab data-structures} 

% typedef struct task_s {
%     uint16_t    tID;     //hardware internal task ID
%     uint64_t    WDPtr;   //address pointing to the task function
%     int         ndeps;   //number of dependencies of the task
%     dep*        deparr;  //list of dependencies of the task
% } task; 

\begin{verbatim}
typedef struct task_s {
    list<dep> predecessors;   // predecessors tasks
    list<dep> successors;     // sucessors tasks
    int       ndeps;          // total number of predecessors
    float     load;           // how long should the task remain on load
} task;

\end{verbatim}

% typedef struct dep_s {
%     uint64_t     varptr;  //address of the dependency variable
%     uint8_t      mode;    //mode of the variable 
% } dep; 

\begin{verbatim}
typedef struct dep_s {
    uint16_t task;      // task that the dep. is heading towards to
    uint8_t  mode;      // mode of the variable from dep.
    uint16_t dID;       // the index of the dep.
} dep;
\end{verbatim}

\guido{Please fill this class in\\}
\isadora{Still needs to define data regarding task trace}

\begin{verbatim}
class TaskLab {
    TaskGraph* tg;
    TaskTrace* tt; (TBD) // to be defined
}
\end{verbatim}

\begin{verbatim}
class TaskGraph {
  private vector<Task> tasks; // tasks structure
  private uint  total_tasks;  // total number of tasks
  private uint  total_deps;   // number of dependencies between tasks
  private uint  dep_range;    // how far a predecessor can be
  private uint  load_time;    // standard load time per task (ms)
  private float max_range;    // max. range standard load time (0 to 1) 
}
\end{verbatim}

\guido{Sometimes you use \& and others *; let us adopt a standard}
\subsection{TaskLab methods}
Below is the list of public and private functions available in TaskLab.

\begin{itemize}
\item \texttt{void TaskGraph::save(const TaskGraph* g, const char* filename)} \newline
This method is responsible for saving a graph as a \textit{.dat} file, by serializing it. It takes as parameter a graph \texttt{g}, where the grpah should be restored and the \texttt{filename} of the serialized graph file.

\item \texttt{void TaskGraph::show(const TaskGraph* g, const char* filename)} \newline
Method responsible for saving a graph as a \textit{.dot} file, allowing to visualize it. It takes as parameter a pointer to the graph to be saved  \texttt{g}, and the corresponding \texttt{filename}  of the graph.

\guido{Would it be better to have the defaults set within the class instead of in the method signature?}
\item \texttt{\noindent void TaskLab::generate \newline 
(TaskGraph** graph, uint n, uint m, uint d, uint t, float r)} \newline
Method responsible for generating the DAG that represents the application with task parallelism to be simulated.  The parameters requires by the grph generation method are described below; notice that all the optional parameters have a default value, which can be defined by the user.
\begin{itemize}
\item \texttt{graph} is a pointer where the final graph will be stored. It must be freed by the user afterwards; 
\item \texttt{n} is the number of tasks to be generated; 
\item \texttt{m} is the maximum number of IN/INOUT dependencies that has to be created on each task; 
\item \texttt{d} sets how far a predecessor may be from a parent (optional); 
\item \texttt{t} is is the standard load time per task (ms) (optional); 
\item \texttt{r} is the maximum range from standard load time (from 0 to 1) (optional). 
\end{itemize}

\item \texttt{void TaskLab::dispatch(TaskGraph* g)} \newline
Method responsible for dispatching a graph \texttt{g} to the MTSP runtime, displaying on-the-fly information regarding the tasks and if it succeeded the execution. The success is defined by checking if the graph was executed in the appropriate order.   The dispatcher   communicates with the runtime by calling its functions in order to describe each of the graph tasks and execute them. Each task triggers an arbitrary function that simply awaits for the given load time.

\guido{The methods below are under specification yet, so please feel free to change it if you want; notice that we might need a file format for the trace}
\item \texttt{void TaskLab::beginTrace(void)} (TBD) \newline
This method starts to capture a task trace from the execution of the program currently running by the runtime.

\item \texttt{void TaskLab::endTrace(void)} (TBD) \newline
This method ends to capture a task trace from the execution of the program currently running by the runtime. It returns a task trace.

\item \texttt{bool TaskLab::runTrace(TaskTrace* t)} (TBD) \newline
This method dispatches a previously task trace \texttt{t} captured from the execution of a particular program and runs it on the runtime. It returns true if the trace executed correctly and false otherwise.
\end{itemize}

\subsection{Task graph generation procedure}
The generation of the task graph works as follows: first, the graph is created with \texttt{n} tasks, specified by the user. For each created task: a random load time is assigned according to the \texttt{default load time} and its \texttt{maximum range} - for example, if the default load time is 1000 ms and the load range is 0.25, it may vary from 750ms to 1250ms; a random number of dependencies is created, ranging from 1 to \texttt{maximum}, also specified by the user.   Then, each dependency in the task is described as follows: a unique id is assigned, for later on purposes; a predecessor that haven't been picked yet is selected, which its \texttt{maximum distance} is defined according to the parameter of dependency range; the predecessor task receives an OUT dependency, with the same id as the successor dependency, relating both tasks; finally, the dependency is assigned as IN or INOUT, randomly. 

\subsection{Task graph validation procedure}
The execution of the graph is validated by creating a boolean list with all the dependencies of the graph based on its unique id, assigned initially to false. It consists of: when a task executes, it assigns all of its successors' dependency as true. Then, if a task executes and all of its predecessors' dependency is defined as true, it means that is safe to execute, i.e. all of its father tasks have already executed; otherwise, the execution is not correct. If every task executed correctly, it means that the graph is validated and successfully executed.

\section{Next Steps}
The v0.2 release of the TâskLab  is currently user development. Some features that it will provide are:
 \begin{itemize}
\item Compatibility with other runtimes;
\item More customization regarding graph generation;
\item Trace capture, storage and replay of task  execution. 
\end{itemize}
\end{document}
